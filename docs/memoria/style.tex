%%% style.tex %%%

%%%%% --------------------------------------------- %%%%%
%	   PACKAGES AND OTHER DOCUMENT CONFIGURATIONS       %
%%%%% --------------------------------------------- %%%%%


\usepackage{amsmath,amsfonts,stmaryrd,amssymb} % Math packages
\usepackage{fancyhdr}
\usepackage{enumerate} % Custom item numbers for enumerations
\usepackage[ruled]{algorithm2e} % Algorithms
\usepackage[framemethod=tikz]{mdframed} % Custom boxed/framed environments

\usepackage{xcolor} % Colours
\definecolor{mGreen}{rgb}{0,0.6,0}
\definecolor{mGray}{rgb}{0.5,0.5,0.5}
\definecolor{mPurple}{rgb}{0.58,0,0.82}
\definecolor{backgroundColour}{rgb}{0.95,0.95,0.92}

\usepackage{listings} % File listings, with syntax highlighting
\lstset{
	basicstyle=\ttfamily, % Typeset listings in monospace font
	extendedchars=true,  % Soporta caracteres extendidos
	literate={├}{{|-}}1 {─}{{-}}1 {●}{{$\bullet$}}1 {└}{{\texttt{\textbackslash}}}1
}
\lstdefinestyle{PythonStyle}{
	backgroundcolor=\color{backgroundColour},   
	commentstyle=\color{mGreen}\textit,
	keywordstyle=\color{magenta}\bfseries,
	numberstyle=\tiny\color{mGray},
	stringstyle=\color{mPurple},
	basicstyle=\fontfamily{pcr}\selectfont\footnotesize,
	breakatwhitespace=false,         
	breaklines=true,
	captionpos=b,                    
	keepspaces=true,                 
	numbers=left,                    
	numbersep=5pt,
	showspaces=false,                
	showstringspaces=false,
	showtabs=false,                  
	tabsize=2,
	language=Python,
	extendedchars=true,
	inputencoding=utf8
}

\lstdefinestyle{BashStyle}{
	backgroundcolor=\color{lightgray!10},
	keywordstyle=\color{blue!80!black}\bfseries,
	stringstyle=\color{red!70!black},         
	basicstyle=\fontfamily{pcr}\selectfont\small,
	breakatwhitespace=false,
	breaklines=true,
	captionpos=b,
	keepspaces=true,
	numbers=none,                  
	showspaces=false,
	showstringspaces=false,
	showtabs=false,
	tabsize=2,
	language=bash,
	extendedchars=true,
	inputencoding=utf8
}

\lstdefinestyle{TextFileStyle}{
	backgroundcolor=\color{lightgray!5},
	basicstyle=\ttfamily\small,
	numbers=none,
	breaklines=true,
	captionpos=b,
	frame=single,
	rulecolor=\color{gray!30},
	showstringspaces=false,
	tabsize=2,
	language={}
}

\usepackage{colortbl} % Para \arrayrulecolor y colorear tablas
\usepackage{alltt}
\usepackage{array, multirow, tabularx, booktabs} % Tables
\usepackage{float}
\usepackage{tikz}
\usetikzlibrary{quotes,babel, positioning, shapes, shapes.geometric, arrows, calc, external, chains, shadows}
% \usetikzlibrary{arrows.meta, positioning, chains, positioning, shapes.geometric, shadows}
\usepackage{aeguill}
\usepackage{tikzpeople}

\usepackage{pgfplots}

\usepackage[linktoc=none]{hyperref} % para enlaces. linktoc deshabilita indice
\hypersetup{
	colorlinks,
	linkcolor={red}, % footnotes
	urlcolor={blue!80!black}, % urls
	pdftitle={Documentación AppChat},
	pdfsubject={TDS},
	pdfauthor={José Salinas Pardo, Hugo Sánchez Martínez}
}

\usepackage[spanish, es-nodecimaldot]{babel} % Paquete de español

\setlength{\arrayrulewidth}{1.5\arrayrulewidth} % Table rule thickness

% Cambiar título del ToC
\addto\captionsspanish{
	\renewcommand{\contentsname}
	{Contenidos}%
}

\usepackage{listingsutf8}    % Soporte para caracteres UTF-8
\usetikzlibrary{shadows,calc}

\usepackage[inline]{enumitem}   
\makeatletter
% This command ignores the optional argument for itemize and enumerate lists
\newcommand{\inlineitem}[1][]{%
	\ifnum\enit@type=\tw@
	{\descriptionlabel{#1}}
	\hspace{\labelsep}%
	\else
	\ifnum\enit@type=\z@
	\refstepcounter{\@listctr}\fi
	\quad\@itemlabel\hspace{\labelsep}%
	\fi}
\makeatother
\parindent=0pt

% some parameters for customization
\def\shadowshift{3pt,-3pt}
\def\shadowradius{6pt}

\colorlet{innercolor}{black!50}
\colorlet{outercolor}{gray!05}

% this draws a shadow under a rectangle node
\newcommand\drawshadow[1]{
	\begin{pgfonlayer}{shadow}
		\shade[outercolor,inner color=innercolor,outer color=outercolor] ($(#1.south west)+(\shadowshift)+(\shadowradius/2,\shadowradius/2)$) circle (\shadowradius);
		\shade[outercolor,inner color=innercolor,outer color=outercolor] ($(#1.north west)+(\shadowshift)+(\shadowradius/2,-\shadowradius/2)$) circle (\shadowradius);
		\shade[outercolor,inner color=innercolor,outer color=outercolor] ($(#1.south east)+(\shadowshift)+(-\shadowradius/2,\shadowradius/2)$) circle (\shadowradius);
		\shade[outercolor,inner color=innercolor,outer color=outercolor] ($(#1.north east)+(\shadowshift)+(-\shadowradius/2,-\shadowradius/2)$) circle (\shadowradius);
		\shade[top color=innercolor,bottom color=outercolor] ($(#1.south west)+(\shadowshift)+(\shadowradius/2,-\shadowradius/2)$) rectangle ($(#1.south east)+(\shadowshift)+(-\shadowradius/2,\shadowradius/2)$);
		\shade[left color=innercolor,right color=outercolor] ($(#1.south east)+(\shadowshift)+(-\shadowradius/2,\shadowradius/2)$) rectangle ($(#1.north east)+(\shadowshift)+(\shadowradius/2,-\shadowradius/2)$);
		\shade[bottom color=innercolor,top color=outercolor] ($(#1.north west)+(\shadowshift)+(\shadowradius/2,-\shadowradius/2)$) rectangle ($(#1.north east)+(\shadowshift)+(-\shadowradius/2,\shadowradius/2)$);
		\shade[outercolor,right color=innercolor,left color=outercolor] ($(#1.south west)+(\shadowshift)+(-\shadowradius/2,\shadowradius/2)$) rectangle ($(#1.north west)+(\shadowshift)+(\shadowradius/2,-\shadowradius/2)$);
		\filldraw ($(#1.south west)+(\shadowshift)+(\shadowradius/2,\shadowradius/2)$) rectangle ($(#1.north east)+(\shadowshift)-(\shadowradius/2,\shadowradius/2)$);
	\end{pgfonlayer}
}

% create a shadow layer, so that we don't need to worry about overdrawing other things
\pgfdeclarelayer{shadow} 
\pgfsetlayers{shadow,main}

\newsavebox\mybox
\newlength\mylen

\newcommand\shadowimage[2][]{%
	\setbox0=\hbox{\includegraphics[#1]{#2}}
	\setlength\mylen{\wd0}
	\ifnum\mylen<\ht0
	\setlength\mylen{\ht0}
	\fi
	\divide \mylen by 120
	\def\shadowshift{\mylen,-\mylen}
	\def\shadowradius{\the\dimexpr\mylen+\mylen+\mylen\relax}
	\begin{tikzpicture}
		\node[anchor=south west,inner sep=0] (image) at (0,0) {\includegraphics[#1]{#2}};
		\drawshadow{image}
\end{tikzpicture}}

\renewcommand{\footnoterule}{%
	\vspace{1cm} % Espacio adicional antes de la línea
	\hrule width 0.8\textwidth % Ajusta el ancho de la línea (80% del ancho del texto)
	\vspace{2cm} % Espacio adicional después de la línea
}


%%%%% --------------------------------------------- %%%%%
%	                 DOCUMENT MARGINS
%%%%% --------------------------------------------- %%%%%

\usepackage{geometry} % Required for adjusting page dimensions and margins

\geometry{
	a4paper, % Papaer size
	margin={2cm,3cm},
	headheight=14pt, % Header height
	footskip=1cm, % Space from the bottom margin to the baseline of the footer
	headsep=1.2cm, % Space from the top margin to the baseline of the header
	%showframe, % Uncomment to show how the type block is set on the page
}

%%%%% --------------------------------------------- %%%%%
%	                      FONTS                         %
%%%%% --------------------------------------------- %%%%%

\usepackage[utf8]{inputenc} % Required for inputting international characters
\usepackage[T1]{fontenc} % Output font encoding for international characters
\usepackage[
protrusion=true,
activate={true,nocompatibility},
final,
tracking=true,
kerning=true,
spacing=true,
factor=1100]{microtype}
\SetTracking{encoding={*}, shape=sc}{40}


% Choose font:
\usepackage{mathptmx} 	 % Times
% \usepackage{mathpazo}  % Palatino
% \usepackage{lmodern}	 % Upgraded LaTeX font

%%%%% --------------------------------------------- %%%%%
%	            COMMAND LINE ENVIRONMENT                %
%%%%% --------------------------------------------- %%%%%

% Usage:
% \begin{commandline}
	%	\begin{verbatim}
		%		$ ls
		%		
		%		Applications	Desktop	...
		%	\end{verbatim}
	% \end{commandline}

\mdfdefinestyle{commandline}{
	leftmargin=10pt,
	rightmargin=10pt,
	innerleftmargin=15pt,
	middlelinecolor=black!50!white,
	middlelinewidth=2pt,
	frametitlerule=false,
	backgroundcolor=black!5!white,
	frametitle={Bash},
	frametitlefont={\normalfont\sffamily\color{white}\hspace{-1em}},
	frametitlebackgroundcolor=black!50!white,
	nobreak,
}

% Define a custom environment for command-line snapshots
\newenvironment{commandline}{
	\medskip
	\begin{mdframed}[style=commandline]
	}{
	\end{mdframed}
	\medskip
}

%%%%% --------------------------------------------- %%%%%
%	             FILE CONTENTS ENVIRONMENT              %
%%%%% --------------------------------------------- %%%%%

% Usage:
% \begin{file}[optional filename, defaults to "File"]
	%	File contents, for example, with a listings environment
	% \end{file}

\mdfdefinestyle{file}{
	innertopmargin=1.6\baselineskip,
	innerbottommargin=0.8\baselineskip,
	innerleftmargin=1.6\baselineskip,
	topline=false, bottomline=false,
	leftline=false, rightline=false,
	leftmargin=2cm,
	rightmargin=2cm,
	singleextra={%
		\draw[fill=black!10!white](P)++(0,-1.2em)rectangle(P-|O);
		\node[anchor=north west]
		at(P-|O){\ttfamily\mdfilename};
		%
		\def\l{3em}
		\draw(O-|P)++(-\l,0)--++(\l,\l)--(P)--(P-|O)--(O)--cycle;
		\draw(O-|P)++(-\l,0)--++(0,\l)--++(\l,0);
	},
	nobreak,
}

% Define a custom environment for file contents
\newenvironment{file}[1][File]{ % Set the default filename to "File"
	\medskip
	\newcommand{\mdfilename}{#1}
	\begin{mdframed}[style=file]
	}{
	\end{mdframed}
	\medskip
}

%%%%% --------------------------------------------- %%%%%
%	          NUMBERED QUESTIONS ENVIRONMENT            %
%%%%% --------------------------------------------- %%%%%

% Usage:
% \begin{question}[optional title]
	%	Question contents
	% \end{question}

\mdfdefinestyle{question}{
	innertopmargin=1.2\baselineskip,
	innerbottommargin=0.8\baselineskip,
	roundcorner=5pt,
	nobreak,
	singleextra={%
		\draw(P-|O)node[xshift=1em,anchor=west,fill=white,draw,rounded corners=5pt]{%
			Question \theQuestion\questionTitle};
	},
}

\newcounter{Question} % Stores the current question number that gets iterated with each new question

% Define a custom environment for numbered questions
\newenvironment{question}[1][\unskip]{
	\bigskip
	\stepcounter{Question}
	\newcommand{\questionTitle}{~#1}
	\begin{mdframed}[style=question]
	}{
	\end{mdframed}
	\medskip
}


%%%%% --------------------------------------------- %%%%%
%	              WARNING TEXT ENVIRONMENT              %
%%%%% --------------------------------------------- %%%%%

% Usage:
% \begin{warn}[optional title, defaults to "Warning:"]
	%	Contents
	% \end{warn}

\mdfdefinestyle{warning}{
	topline=false, bottomline=false,
	leftline=false, rightline=false,
	nobreak,
	singleextra={%
		\draw(P-|O)++(-0.5em,0)node(tmp1){};
		\draw(P-|O)++(0.5em,0)node(tmp2){};
		\fill[black,rotate around={45:(P-|O)}](tmp1)rectangle(tmp2);
		\node at(P-|O){\color{white}\scriptsize\bf !};
		\draw[very thick](P-|O)++(0,-1em)--(O);%--(O-|P);
	}
}

% Define a custom environment for warning text
\newenvironment{warn}[1][Warning:]{ % Set the default warning to "Warning:"
	\medskip
	\begin{mdframed}[style=warning]
		\noindent{\textbf{#1}}
	}{
	\end{mdframed}
}


%%%%% --------------------------------------------- %%%%%
%                 INFORMATION ENVIRONMENT               %
%%%%% --------------------------------------------- %%%%%

% Usage:
% \begin{info}[optional title, defaults to "Info:"]
	% 	contents
	% 	\end{info}

\mdfdefinestyle{info}{%
	topline=false, bottomline=false,
	leftline=false, rightline=false,
	nobreak,
	singleextra={%
		\fill[black](P-|O)circle[radius=0.4em];
		\node at(P-|O){\color{white}\scriptsize\bf i};
		\draw[very thick](P-|O)++(0,-0.8em)--(O);%--(O-|P);
	}
}

% Define a custom environment for information
\newenvironment{info}[1][Info:]{ % Set the default title to "Info:"
	\medskip
	\begin{mdframed}[style=info]
		\noindent{\textbf{#1}}
	}{
	\end{mdframed}
}

